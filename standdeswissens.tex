Grundsätzlich unterscheidet man zwischen dem Begriff "'Digitale Signatur"' als Verfahren zur Verschlüsselung von Daten und "'Elektronische Unterschrift"' als Verfahren zur Erfassung der Daten für das Verschlüsselungsverfahren. Unter "'Biometrischer Unterschrift"' versteht man eine "'Elektronische Unterschrift"' verknüpft mit biometrischen Daten des Unterschriftgebers.
\newline
Die Gesetze IuKDG und SigG sind zum Schutz der Anwender und Nutzer geschaffen wurden. Diese dienen zur Gewährleistung von Vertraulichkeit, Verfügbarkeit und Zurechenbarkeit der Teilnehmer am Signaturverfahren. \cite{standdeswissens1} Als Hauptaufgabe des SigG zählt die Prüfung von Integrität und Authentizität der Kommunikationspartner. Die Garantie der Unversehrtheit des Dokuments und des Inhalts können als weitere Ziele genannt werden. Die "'Elektronische Unterschrift"' dient ebenfalls als Beweismittel vor Gericht. Im Signaturverfahren wird zwischen den nachfolgenden vier Beteiligten unterschieden. \cite{standdeswissens2}