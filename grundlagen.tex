Die ersten Überlegungen und gesetzlichen Regelungen zum Thema biometriche Merkmale begannen bereits durch
das sogenannte Volkszählungsurteil des Bundesverfassungsgerichts vom 15. Dezember 1983, welches besagt, dass
die Bürger grundsätzlich ein Selbstbestimmungsrecht über die Vergabe und Verwendung ihrer Daten besitzen.
Dies gilt als Meilenstein des Datenschutzes und legt das informationelle Selbstbestimmungsrecht als
Grundrecht gemäß Art. 2 Abs. 1 iVm. Art. 1 Abs. 1 des Grundgesetzes sowie auf europäischer Ebene mit Art. 8
Abs. 1 der Europäischen Menschenrechtskonvention fest. \cite{grundlagen1}\cite{grundlagen2}\cite{grundlagen3} 
Mit dem Informations - und Kommunikationsdienste - Gesetz (IuKDG)\footnote{\label{foot:2}BGBl. I S. 1870,
1872 ff. vom 22. Juli 1997 \cite{grundlagenFN1}} wurden 1997 die ersten gesetzlichen Regelungen zu elektronischen
Signaturen veröffentlicht. Im Jahr 2001 ist das Gesetz über Rahmenbedingungen für elektronische Signaturen
(SigG)\footnote{\label{foot:3}BGBl. I S. 876. vom 16. Mai 2001 \cite{grundlagenFN2}} als Nachfolger des
IuKDG in Kraft getreten und wird stetig anhand neuer gesetzlicher Regelungen aktualisiert. \cite{grundlagen4}