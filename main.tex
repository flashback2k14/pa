\documentclass[a4paper, 12pt]{article}
%--------------------------------------
% Abstände
%--------------------------------------
\usepackage[left=30mm,right=20mm,top=25mm,headsep=15mm]{geometry}
\usepackage[onehalfspacing]{setspace}
%--------------------------------------
% encoding
%--------------------------------------
\usepackage[utf8]{inputenc}
\usepackage[T1]{fontenc}
%--------------------------------------
% German-specific commands
%--------------------------------------
\usepackage[ngerman]{babel}
%--------------------------------------
% Hyphenation rules
%--------------------------------------
\usepackage{hyphenat}
\hyphenation{Mathe-matik wieder-gewinnen} %Überflüssig?
%--------------------------------------
% Hyperlinks
%--------------------------------------
\usepackage{url}
\usepackage{hyperref}
%--------------------------------------
% Grafiken
%--------------------------------------
\usepackage[rightcaption]{sidecap}
\usepackage{graphicx}
%--------------------------------------
% Kopf-& Fußzeile
%--------------------------------------
\usepackage{color}
\definecolor{fzp}{rgb}{0.565,0.612,0.671}
\usepackage{fancyhdr}
\setlength{\headheight}{37.9pt}
\pagestyle{fancy}
\fancyhf{}
%Kopfzeile links bzw. innen
\fancyhead[L]{\small{Leipzig - Ettlingen - Passau}}
%Kopfzeile mittig
\fancyhead[C]{}
%Kopfzeile rechts bzw. außen
\fancyhead[R]{\includegraphics{FZP_RGB2.png}}
%Linie oben
%\renewcommand{\headrulewidth}{2pt}
\renewcommand\headrule
{{\color{fzp}%
	\hrule height 3pt width \headwidth
	\vspace{1pt}%
	\hrule height 1pt width \headwidth
	\vspace{-4pt}
}}
%Fußzeile rechts
\fancyfoot[R]{\thepage}
%Linie unten
%\renewcommand{\footrulewidth}{0.5pt}
%--------------------------------------
% Quellenverzeichnis
%--------------------------------------
\usepackage[backend=biber,defernumbers,sorting=none,firstinits=true]{biblatex}
\addbibresource{biblo.bib}
%---------------------------------------------------
% SyntaxHighlighting
%--------------------------------------
\usepackage{listings}

\definecolor{codegreen}{rgb}{0,0.6,0}
\definecolor{codegray}{rgb}{0.5,0.5,0.5}
\definecolor{codepurple}{rgb}{0.58,0,0.82}
\definecolor{backcolour}{rgb}{0.95,0.95,0.92}
 
\lstdefinestyle{mystyle}{
    backgroundcolor=\color{backcolour},   
    commentstyle=\color{codegreen},
    keywordstyle=\color{magenta},
    numberstyle=\tiny\color{codegray},
    stringstyle=\color{codepurple},
    basicstyle=\footnotesize,
    breakatwhitespace=false,         
    breaklines=true,                 
    captionpos=b,                    
    keepspaces=true,                 
    numbers=left,                    
    numbersep=5pt,                  
    showspaces=false,                
    showstringspaces=false,
    showtabs=false,                  
    tabsize=2
}
 
\lstset{style=mystyle}
%--------------------------------------
\begin{document}
%--------------------------------------
% großzügige Formatierungsweise
%--------------------------------------
\sloppy
%--------------------------------------
% Deckblatt
%--------------------------------------
\thispagestyle{empty}
\begin{center}
\Large{Berufliche Studienakademie Sachsen}
\end{center}
 
\begin{center}
\Large{Fachbereich Informatik}
\end{center}
\hspace{4cm}

\begin{center}
\textbf{\LARGE{Praxisarbeit}}
\end{center}
\hspace{4cm}

\begin{flushleft}
\begin{tabular}{l p{10pt} p{290pt}}
& & \textbf{im Studiengang Informatik}\\
\\
\textbf{Thema:} & &  Biometrische Unterschrift in einem mittelständischen Unternehmen 
der Vermietungsbranche zur personalisierten Warenabgabe\\
& & \\
& & \\
& & \\
\textbf{eingereicht von:} & & Sebastian Kloppe \\
& &                           Braustraße 17 - 19\\
& &                           04107 Leipzig\\
& & \\
\textbf{Seminargruppe:} & & CS13-2 \\
\textbf{Matrikelnummer:} & & 5000349\\
& & \\
& & \\
\textbf{Betreuer:} & & B.Sc. Benjamin Teßmar\\
& &                    Funk, Zander \& Partner GmbH\\
& &                    Kreuzstraße 7\\
& &                    04103 Leipzig\\
& & \\
& & \\
& & \\
& & \\
& & \\
& & \\
& & \\
\textbf{eingereicht am:} & & 11. Juli 2014 in Leipzig\\
\end{tabular}
\end{flushleft}
%--------------------------------------
% Inhaltsverzeichnis
%--------------------------------------
\newpage
\thispagestyle{plain}
\pagenumbering{Roman}
\tableofcontents
%--------------------------------------
% Abbildungsverzeichnis
%--------------------------------------
\newpage
\thispagestyle{plain}
\listoffigures

\newpage
%--------------------------------------
% Einleitung
%--------------------------------------
\clearpage 
\pagenumbering{arabic}   

\section{Einleitung}
\label{sec:Einleitung}

Das Unternehmen, die Funk, Zander \& Partner GmbH, hat sich seit der Gründung im Jahr 1992 auf die elektronische Datenverarbeitung und Unternehmensberatung spezialisiert. Das weitere Unternehmensangebot umfasst anwenderbezogene Schulungen und Programmierungsarbeiten. Das Thema "'Biometrische Unterschrift"' ist im Rahmen einer Softwareumstellung eines Kunden entstanden. Aus diesem Grund wurde am 16. April 2014 bei Herrn Prof. Dr. Brunner das oben genannte, abweichende Thema für die Praxisarbeit beantragt und genehmigt.


\subsection{Motivation}
\label{subsec:Motivation}

Unser Auftraggeber ist ein inhabergeführtes, mittelständisches Unternehmen und angewiesen, aufgrund zunehmender Komplexität der Unternehmensprozesse sowie steigender Auftragszahlen, dass bestehende System zur Auftrags- und Materialverwaltung sowie der Kundendatenverwaltung auf Grundlage von MS-DOS durch die Sage Office Line, einer vollständigen ERP-Lösung\footnote{\label{foot:1}Enterprise-Resource-Planning: Effizente Planung / Steuerung von Unternehmensressourcen. \cite{ERP}}, der Firma Sage Software GmbH (Sage) sowie durch Zusatzmodule der Firma Funk, Zander \& Partner GmbH (FZP) abzulösen, um wettbewerbsfähig zu bleiben. \cite{einleitung1}

\subsection{Zielsetzung}
\label{subsec:Zielsetzung}

Im Zuge der Softwareumstellung wird angestrebt die Warenabgabe mittels biometrischer Signatur auf Lieferscheinen und Rechnungen zu personalisieren. Darüber hinaus entsteht ein Nachweis über die erbrachten Lieferungen und Leistungen. Hierfür wird eine Schnittstellenlösung von FZP mit Hilfe eines Signatur-Tablets der Firma Wacom entwickelt. Die neu geschaffene Schnittstelle reicht die erfassten Daten an eine bestehende Schnittstelle von FZP zur Erstellung von Vermietungsbelegen sowie für die Erstellung von Auswertungen an die Sage Office Line weiter. \cite{einleitung1}
%--------------------------------------
% Grundlagen
%--------------------------------------
\section{Grundlagen}
\label{sec:Grundlagen}

Die ersten Überlegungen und gesetzlichen Regelungen zum Thema biometriche Merkmale begannen bereits durch
das sogenannte Volkszählungsurteil des Bundesverfassungsgerichts vom 15. Dezember 1983, welches besagt, dass
die Bürger grundsätzlich ein Selbstbestimmungsrecht über die Vergabe und Verwendung ihrer Daten besitzen.
Dies gilt als Meilenstein des Datenschutzes und legt das informationellen Selbstbestimmungsrecht als
Grundrecht gemäß Art. 2 Abs. 1 iVm. Art. 1 Abs. 1 des Grundgesetzes sowie auf europäischer Ebene mit Art. 8
Abs. 1 der Europäischen Menschenrechtskonvention fest. \cite{grundlagen1}\cite{grundlagen2}\cite{grundlagen3} 
Mit dem Informations - und Kommunikationsdienste - Gesetz (IuKDG)\footnote{\label{foot:2}BGBl. I S. 1870,
1872 ff. vom 22. Juli 1997 \cite{grundlagenFN1}} wurden die ersten gesetzlichen Regelungen zu elektronischen
Signaturen veröffentlicht. Im Jahr 2001 ist das Gesetz über Rahmenbedingungen für elektronische Signaturen
(SigG)\footnote{\label{foot:3}BGBl. I S. 876. vom 16. Mai 2001 \cite{grundlagenFN2}} als Nachfolger des
IuKDG in Kraft getreten. \cite{grundlagen4}

\subsection{Stand des Wissens}
\label{sec:Stand des Wissens}

Die Gesetze IuKDG sowie SigG sind zum Schutz und zur Gewährleistung von Vertraulichkeit, Integrität, Authentizität, Verfügbarkeit sowie Zurechenbarkeit der Anwender und Nutzer geschaffen wurden. \cite{standdeswissens1} Als Hauptaufgabe des SigG zählt die Identifizierung der Kommunikationspartner. Weitere Ziele sind die Garantie der Unversehrtheit des Dokuments und des Inhalts sowie die Beweiskraft vor Gericht. \cite{standdeswissens2}


\subsubsection{Nutzer / Anwender}
\label{sec:Nutzer / Anwender}

Laut \S 2 Nr. 9 SigG kann ein Nutzer bzw. Antragsteller auf die Ausfertigung einer digitalen Signatur nur eine natürliche Person sein. Eine juristische Person in Form einer Gesellschaft mit beschränkter Haftung (GmbH) oder einer Aktiengesellschaft (AG) darf dadruch nicht als Nutzer von digitalen Signaturen direkt auftreten. Der Grund hierfür ist, dass eine GmbH oder eine AG durch mehrere natürliche Personen, wie dem Aufsichtsrat, Vorstand oder mehreren Geschäftsführern, vertreten werden kann. Eine eindeutige Identifikation und Authentikation des Nutzers der digitalen Signatur ist infolgedessen nicht mehr möglich. Andererseits besteht die Möglichkeit, dass eine natürliche Person die Nutzung seines Signaturschlüssels für weitere natürliche Personen autorisiert und entsprechend des Signaturgesetzes beispielsweise mehrere Geschäftsführer einer GmbH mit einem Signaturschlüssel auftreten können. Des Weiteren ist es dem Signaturschlüssel-Inhaber möglich zusätzliche Informationen gemäß \S 5 Abs. 2 SigG in Form eines Attribut-Zertifikates zu hinterlegen. In einem Attribut-Zertifikat können Vertretungsvollmachten für dritte Personen, berufsrechtliche Zulassung und sonstige Angaben zur Person eingetragen werden. Außerdem besteht eine Verpflichtung gegenüber dem Nutzer von digitalen Signaturen, den privaten Schlüssel sorgfältig aufzubewahren. Die Aufbewahrung ist wichtig um den Gebrauch und die damit einhergehende Identitätsannahme des Signaturschlüssel-Inhabers von unbefugten Personen zu verhindern. Der private Schlüssel ist meistens auf einer Chipkarte hinterlegt und vor Veränderungen geschützt. \cite{standdeswissens3}\cite{standdeswissens4}

\subsubsection{Zertifizierungstelle / Trustcenter}
\label{sec:Zertifizierungstelle / Trustcenter}

Laut \S 2 Nr. 8 SigG können Zertifizierungsstellen von natürlichen und juristischen Personen betrieben werden. Der Betrieb einer Zertifizierungsstelle ist grundsätzlich genehmigungsfrei, obliegt dennoch folgenden Bestimmungen nach \S 4 SigG. Der Betrieb einer Zertifizierungsstelle setzt Fachkunde, Zuverlässigkeit, Deckungsvorsorge gemäß \S 12 SigG gegen Schadenersatzansprüche voraus. Des Weiteren ist die Erfüllung der Sicherheitsansprüche anhand eines von der Prüfstelle abgenommenen Sicherheitskonzepts notwendig. "'Die erforderliche Fachkunde liegt vor, wenn die im Betrieb eines Zertifizierungsdienstes tätigen Personen über die für diese Tätigkeit notwendigen Kenntnisse, Erfahrungen und Fertigkeiten verfügen"' \footnote{\S 4 Abs. 2 Satz 3 Signaturgesetz \cite{grundlagenFN2}}. "'Die erforderliche Zuverlässigkeit besitzt, wer die Gewähr dafür bietet, als Zertifizierungsdiensteanbieter die für den Betrieb maßgeblichen Rechtsvorschriften einzuhalten"' \footnote{\S 4 Abs. 2 Satz 2 Signaturgesetz \cite{grundlagenFN2}}. Als Trust Center bezeichnet man die Einrichtungen, die die Verschlüssung von Daten anbieten und das Hochsicherheitsrechenzentrum betreiben. Zertifizierungsstellen und Trust Center werden benötigt, wenn eine große Anzahl von Benutzern am digitalen Signaturverfahren teilnehmen. Die Hauptaufgabe der Zertifizierungsstelle ist die Identifizierung der Antragssteller für einen Signaturschlüssel mittels Personalausweis sowie die Vergabe und Zuordnung der öffentlichen Signaturschlüssel. Des Weiteren sichern sie die Authentizität des Absenders im Rechtsverkehr, vergleichbar mit einer amtlichen Beglaubigung von Behörden und Notaren. \cite{standdeswissens3}\cite{zertstelle1} 
\begin{figure}[!ht]
    \centering
    \includegraphics[height=250pt, width=300pt]{trustcenterNeu3.jpg}
    \caption[Schema einer Zertifizierungsstelle]{\small{Schema einer Zertifizierungsstelle \cite{trust1}}}
\end{figure}
\newline
\pagebreak
\textbf{} %Workaround
\newline
Wie auf der Abbildung 1 zu sehen ist, lässt sich die Zertifizierungsstelle in zwei Bereiche einteilen. Auf der einen Seite steht die Annahmestelle, welche die Identifikation der Antragsteller mit Personalausweis und weiteren notariell beglaubigten Dokumenten, z.B. Zulassungen übernehmen. Auf der anderen Seite steht die Zertifizierungsstelle, welche die Erzeugung der Schlüsselpaare und Zertifikate übernimmt. Die Zertifizierungsstelle muss sicherstellen, dass niemand den privaten Schlüssel modifizieren kann. Weiterhin muss der Dienst zur Prüfung der digitalen Signaturen immer verfügbar sein. Abgesehen davon muss sichergestellt werden, dass nur der Antragsteller den privaten Schlüssel erhält. Dies wird meistens mit Chipkarten realisiert. Auf den Chipkarten befindet sich der private Schlüssel und ist vor Auslesung, Manipulation oder Löschung geschützt. Zusätzlich müssen Zertifizierungsstellen noch Dienste, wie die Zulassung von Pseudonyme zur Wahrung des Personlichkeitsrechts des Schlüsselinhabers und ein 24-Stunden-Sperrdienst für Signaturschlüssel anbieten. Des Weiteren muss die Schlüsselaufbewahrung vom Auftraggeber ausdrücklich erklärt werden, ansonsten ist die Speicherung von privaten Schlüsseln in Zertifizierungsstellen nicht erlaubt. Die Pflege des Schlüsselverzeichnis ist von besonderer Bedeutung, da jederzeit die Identifikation von Kommunikationspartnern möglich sein muss. Das Schlüsselverzeichnis muss ständig aktuell und funktionstüchtig sein. Ein weiterer Dienst ist der Zeitstempeldienst. Dieser dient zur genauen Festhaltung, wann ein Dokument erzeugt, verändert oder gelöscht wurde, mittels Zeitangabe der Zertifizierungsstelle im Hashwert für das signierte Dokument. Anwendung findet dieses Verfahren bei der Einhaltung von gesetzlichen Fristen und der Festlegung von Gültigkeitsdauern. \cite{standdeswissens3}\cite{zertstelle1}

\subsubsection{Pr"ufstelle}
\label{sec:Pr"ufstelle}

Die Prüfstelle ist zur Prüfung des Sicherheitskonzeptes der Zertifizierungsstelle beauftragt und benötigt die Anerkennung der Regulierungsbehörde. Die Aufgabe der Prüfstelle ist die Kontrolle und Abnahme der Hard- und Softwarekomponeten gemäß dem Signaturgesetz. Der Prüfbericht wird anschließend von der Regulierungsbehörde veröffentlicht. \cite{standdeswissens3}

\subsubsection{Regulierungsbeh"orde}
\label{sec:Regulierungsbeh"orde}

Die Regulierungsbehörde, auch Bundesnetzagentur (BNetzA) genannt, ist eine Bundesoberbehörde mit Sitz in Bonn, welche am 01.01.1998 als Nachfolgerin des Bundesministeriums für Post und Telekommunikation sowie der Deutschen Bundespost gegründet wurde. Die Kontrolle der Einhaltung der gesetzlichen Bestimmungen des Signaturgesetzes ist nur ein Bestandteil der Behörde, welche auch für das Wettbewerbsrecht der Telekommunikationsbranche zuständig ist. Die Hauptaufgaben der BNetzA im Signaturprozess liegt in der Vergabe und Entziehung der Genehmigungen für den Betrieb einer Zertifizierungsstelle, der Kontrolle der Zertifizierungsstellen sowie der Ausstellung eines Zertifikates, das die Zertifizierungsstelle authentisiert. \cite{standdeswissens3}\cite{regBeh1}

\subsection{Technologien}
\label{sec:Technologien}

Die wesentlichen Schritte im digitalen Signaturgsprozess sind signieren, verschlüsseln und prüfen mittels des asymmetrischen Schlüssel Konzepts. Das asymmetrische Schlüssel Konzept besteht aus der Generierung eines privaten Schlüssels, dieser wird in einer Zertifizierungsstelle hinterlegt, und einem privaten Schlüssel, diesen besitzt der Signaturschlüssel-Inhaber, welcher auf einer Chipkarte vor Manipulationen geschützt ist und nicht ausgelesen werden kann. \cite{techno1} 

\subsubsection{Erstellung Dokumente}
\label{sec:Erstellung Dokumente}

Das asymmetrische Schlüsselkonzept geht auf das von Whitfield Diffie und Martin Hellmann im Jahr 1976 entwickelte Verschlüsselungsverfahren DH76 zurück. In dem Verfahren werden zwei voneinander abhängige Schlüssel generiert. Ein Schlüssel dient zur Verschlüsselung und ist geheim. Der andere Schlüssel wird zur Entschlüsselung verwendet und ist öffentlich. Die digitale Signatur dient vorrangig nicht der Unkenntlichmachung der Dokumente und Information, sondern der Feststellung, ob die signierten Informationen unversehrt bei dem Empfänger angekommen sind oder verändert wurden. Darüber hinaus wird es zur Identifizierung des Signaturschlüssel-Inhabers verwendet. Die Prüfung auf Echtheit des Signaturschlüssels wird mit dem öffentlichen Schlüssel übernommen. Der private Schlüssel signiert nicht das gesamte Dokument, sondern nur einen repräsentativen Bereich. \cite{techno1}
\begin{figure}[!ht]
    \centering
    \includegraphics[width=\textwidth]{ErstellungAbsender2.jpg}
    \caption[Erstellung eines Dokuments mit digitaler Signatur]{\small{Erstellung eines Dokuments mit digitaler Signatur. \cite{techno3}}}
    \label{fig:2}
\end{figure}\\
Der erste Schritt im Signierungsprozess ist das Erzeugen eines Fingerabdrucks (Hashwert) des Dokuments. Der Fingerabdruck wird mittels Hash-Algorithmus von einem Teil des Dokuments erstellt und ist mit hoher Wahrscheinlichkeit eindeutig. "'Ein Hash-Algorithmus ist eine mathematische Funktion, die eine beliebig große Menge an Eingabewerten möglichst gleichmäßig auf eine eingeschränkte Ausgabemenge abbildet"' \cite{techno2}. Eine absolute Eindeutigkeit des Fingerabdrucks kann auf Grund des Zufallsprinzips bei Hash-Algorithmen nicht gewährleistet werden. Anhand des Fingerabdrucks kann nicht auf das Dokument und deren Inhalt geschlossen werden. Weiterhin ist die Erstellung des Dokuments aus dem Fingerabdruck nicht möglich, da der Hashwert von einem beliebigen Teil des Dokuments erstellt wird. Im zweiten Schritt des Prozesses wird der Fingerabdruck mittels privaten Signaturschlüssels zur Sicherstellung der Unversehrtheit des Dokuments verschlüsselt. Im letzten Schritt werden der verschlüsselte Fingerabdruck und das Dokument zu einer Datei vereint. \cite{techno1}   

\subsubsection{Biometrische Daten}
\label{sec:Biometrische Daten}

Neben der Erstellung eines Fingerabdrucks mittels Hash-Algorithmus, können der digitalen Signatur noch weitere Daten, wie biometrische Charakteristika des Signaturschlüssel-Inhabers angehängt werden. Die Biometrie setzt sich aus den Worten "bios" (Leben) und "métron" (Maßstab) zusammen und beschäftigt sich mit Messungen an Lebewesen und den dazu erforderlichen Mess- und Auswertungsverfahren. \cite{bioMet1} Ziel ist die Identifizierung sowie die Authentifizierung von Menschen anhand deren einzigartiger sowie konstanter biometrischer Merkmale. Unter biometrischen Merkamelen versteht man unteranderem, dass Papillarmuster der Finger, die Handgeometrie und die Handlinien, die Vermessung des Gesichtes, die Iris- und Retinastruktur, die DNA, der Stimmabgleich, das Bewegungsmuster sowie die Unterschrift. Bei einer digitalen Signatur mit biometrischen Merkmalen benötigt man hauptsächlich die Schreibgeschwindigkeit, den ausgeübten Druck, den Bereich und die Dauer der Unterschrift sowie das Schriftbild zur Identifizierung und Authentifizierung einer Person. Jedoch ist eine 100 prozentige Genauigkeit bei biometrischen Merkmalen nicht gewährleistet. Zu Fehlern in den Mess- und Auswertungsverfahren kann es durch altersbedingte Veränderungen, Körperveränderungen, wie z. B. Verletzung oder Verlust von Gliedmaßen, der Veränderung der Iris sowie durch Täuschung oder Manipulation von Prüfungsgeräte kommen. \cite{bioMet2}\cite{bioMet3}

\subsubsection{Versendung Dokumente}
\label{sec:Versendung Dokumente}

Für die Versendung des Dokuments werden nach der Vereinigung der Komponenten der Dokumentendatei zusätzlich der öffentliche Signaturschlüssel und das Zertifikat des Signaturschlüssel-Inhabers angehängt und versendet. Anhand des Zertifikates kann die Identitätsprüfung des Absenders durchgeführt werden. Die Bestandteile des Zertifikates sind der Name bzw. das Pseudonym des Signaturschlüssel-Inhabers, die Zertifizierungsstelle und deren Identifizierungsnummer, das Ausstellungsdatum sowie die Gültigkeitsdauer und der öffentliche Signaturschlüssel mit dem verwendeten Hash-Algorithmus. \cite{techno1}\cite{techno4}
\begin{figure}[!ht]
    \centering
    \includegraphics[height=200pt, width=270pt]{versand_doc.PNG}
    \caption[Aufbau versandfertiges Dokument]{\small{Aufbau versandfertiges Dokument \cite{techno6}}}
\end{figure}

\subsubsection{Pr"ufung Dokumente}
\label{sec:Pr"ufung Dokumente}

Die Prüfung des Dokuments erfolgt im ersten Schritt auf Echtheit des öffentlichen Schlüssels mittels der Identifizierungsnummer und des Namens der Zertifizierungsstelle aus dem übermittelten Zertifikat. Im nächsten Schritt wird die Prüfung des Zertifikates mithilfe des öffentlichen Schlüssels aus der öffentlich, zugänglichen Datenbank der Zertifizierungsstelle durchgeführt. Ist die Echtheit sichergestellt wird die Entschüsselung der Signatur durch den öffentlichen Schlüssel durchgeführt. Das Ergebnis der Entschlüsselung ist der vom Absender erzeugte Fingerabdruck. Als letzter Schritt wird anhand des im Zertifkat genannten Hash-Algorithmus erneut ein Fingerabruck des Dokuments erstellt und mit dem entschlüsselten Fingerabdruck verglichen. Die Unversehrheit des Dokuments ist gewährleistet, wenn die beiden Fingerabdrücke des Dokuments übereinstimmen. \cite{techno1}
\begin{figure}[!ht]
    \centering
    \includegraphics[width=\textwidth]{PruefungEmpfaenger2.jpg}
    \caption[Prüfung eines Dokuments mit digitaler Signatur]{\small{Prüfung eines Dokuments mit digitaler Signatur. \cite{techno3}}}
    \label{fig:3}
\end{figure}
%--------------------------------------
% Konzeption
%--------------------------------------
%\newpage
\section{Konzeption}
\label{sec:Konzeption}

\subsubsection*{Einleitung}
Das mittelständige Unternehmen ist in den Bereichen Maschinenbau, Reparaturen und Instandsetzungen sowie Verkauf und Vermietung tätig. Die Produkte des Verkaufs und der Vermietung sind hochwertige Elektromaschinen sowie Werkzeuge und Verbrauchsmaterialien. Desweiteren werden Instandhaltungs- und Reparaturaufträge für Maschinen sowie Spezialanfertigung im Bereich Maschinenbau angeboten. Die Kundschaft besteht überwiegend aus selbstständigen Handwerkern, kleineren Betrieben und Angehörigen aus Land- und Forstwirtschaft.
\subsubsection*{Ist-Zustand}
Die Warenabgabe erfolgt mittels handgeschriebener Rechnungen und Aufträge. Im nächsten Schritt werden die Aufträge und Rechnungen zur weiteren Verwendung in das bestehende MSDOS-System wiederholt händisch übertragen. Im darauffolgenden Schritt werden die handgeschriebenen Belege an die Buchhaltungsabteilung zur Prüfung des Rechnungseingangs und Rechnungsausgangs sowie an die Fertigungsabteilung zur Auftragsbearbeitung weitergereicht.
\subsubsection*{Ziel}
Das Hauptziel der Umstellung ist die Vereinfachung und Beschleunigung der Geschäftsprozesse, Entlastung der Mitarbeiter sowie der Kostenreduzierung und Effizienzsteigerung. Das Ziel der personalisierten Warenabgabe ist der eindeutige Nachweis über die erbrachten Lieferungen und Leistungen sowie der rechtlichen Absicherung gegenüber Dritten. \cite{einleitung1}
\subsubsection*{Umsetzung}
Im weiteren Verlauf des Konzeptes werde ich nur auf die Umsetzung der personalisierten Warenabgabe mittels digitaler Signatur eingehen. 

Im ersten Schritt wurde die Anbindung der neu zuschaffenden Schnittstelle durch FZP zur Erfassung der Daten aus dem Signatur-Tablet und Weitergabe an das bestehende Datenbankschema der Sage Office Line geprüft. Das Ergebnis der Prüfung ist die problemlose Einbindung in das bestehende Datenbankschema sowie die Schaffung einer neuen Datenbanktabelle. Die Anbindung der Schnittstelle zur Sage Office Line ist ebenfalls problemlos möglich mittels DCM \textbf{Erklärung folgt}. DCM heißt \textbf{XXX} und ist eine von Sage geschaffene Methode um externe Schnittstellen mit der Sage Office Line zu verbinden...\\\\
Im zweiten Schritt wurden verschiedene Anbieter von Signatur-Tablets geprüft. Hier hat sich die Firma Wacom schnell heraus kristallisiert, wegen der Unternehmenserfahrungen seit 1983 sowie der guten Anbindungen des grafischen Tablets mittels des bereit gestellten SDK\footnote{\label{foot:4} Software Development Kit: Sammlung von Werkzeugen und Anwendungen, um eine Software zu erstellen, meist inklusive Dokumentation. \cite{SDK}}. \cite{konzept1}
%--------------------------------------
% Ergebnisse
%--------------------------------------
\section{Ergebnisse}
\label{sec:Ergebnisse}

Das Ergebnis der Datenintegrationsprüfung ist die problemlose Einbindung in das bestehende Datenbankschema. Die Anbindung der Schnittstelle zur Sage Office Line erfolgt mittels DCM-Verfahren sowie die Integration der Formulare in die bestehende Benutzeroberfläche der Sage Office Line ist aufgrund des modularen Aufbaukonzepts von Sage problemlos möglich. DCM heißt ""DLL Common Methode"" und ist eine von Sage geschaffene Methode um externe Schnittstellen mit der Sage Office Line zu verbinden... \textbf{Erklärung DCM + Screenshots + Listings}
%--------------------------------------
% Zusammenfassung
%--------------------------------------
\section{Zusammenfassung}
\label{sec:Zusammenfassung}

\Large{Zusammenfassung} 
\newline
content following soon!

this is from DroidEdit on Android to Github. :-)

this is from Github to DroidEdit on Android. :-)

%--------------------------------------
% Selbststänigkeitserklärung
%--------------------------------------
\newpage
\section{Selbstständigkeitserklärung}
\input{selbststaendig}
%--------------------------------------
% Quellenverzeichnis
%--------------------------------------
\newpage
%--------------------------------------
% großzügige Formatierungsweise
%--------------------------------------
%\sloppy
\section{Quellenverzeichnis}
%\renewcommand{\refname}{}

\begin{thebibliography}{99}
   \small{\bibitem {ERP}\url{http://de.wikipedia.org/wiki/Enterprise-Resource-Planning}, Zugriff 12.05.2014}
   \small{\bibitem {grundlagen1} \url{https://www.datenschutzzentrum.de/projekte/biometrie/biometkk.htm}, Zugriff 21.05.2014}
   \small{\bibitem {grundlagen2} \url{http://de.wikipedia.org/wiki/Volksz\%C3\%A4hlungsurteil}, Zugriff 21.05.2014}
   \small{\bibitem {grundlagen3} \url{http://de.wikipedia.org/wiki/Informationelle_Selbstbestimmung}, Zugriff 21.05.2014}
   \small{\bibitem {grundlagenFN1} \url{http://www.bgbl.de/banzxaver/bgbl/start.xav?startbk=Bundesanzeiger_BGBl\&jumpTo=bgbl197s1870.pdf}, 
   Zugriff 21.05.2014}
   \small{\bibitem {grundlagenFN2} \url{http://www.bgbl.de/banzxaver/bgbl/start.xav?startbk=Bundesanzeiger_BGBl\&jumpTo=bgbl101s0876.pdf}, 
   Zugriff 21.05.2014}
   \small{\bibitem {grundlagen4} \url{http://de.wikipedia.org/wiki/Signaturgesetz_\%28Deutschland\%29}\\Zugriff 21.05.2014}
   \small{\bibitem{zertstelle1} \url{http://www.gesetze-im-internet.de/sigg_2001/BJNR087610001.html#BJNR087610001BJNG000300000}, 
   Zugriff 22. Mai 2014}
   \small{\bibitem{techno2} \url{http://wiki.ubuntuusers.de/Hashfunktionen}, Zugriff 28. Mai 2014}
   \small{\bibitem {regBeh1}\url{http://de.wikipedia.org/wiki/Bundesnetzagentur}, Zugriff 27.05.2014}
   \small{\bibitem {techno3}\url{http://commons.wikimedia.org/wiki/File:Electronic-signature-scheme-german.svg}, Zugriff 27.05.2014}
   \small{\bibitem{standdeswissens1}Andreas Bertsch: \textit{Digitale Signaturen}, Seite 3 bis 4, Springer Verlag Deutschland (2002)}
   \small{\bibitem{standdeswissens2}F.Bitzer/K.M.Brisch: \textit{Digitale Signatur}, Seite 21, Springer Verlag Deutschland (1999)}
   \small{\bibitem{standdeswissens3}F.Bitzer/K.M.Brisch: \textit{Digitale Signatur}, Seite 25 bis 37, Springer Verlag Deutschland (1999)}
   \small{\bibitem{standdeswissens4}Andreas Bertsch: \textit{Digitale Signaturen}, Seite 18, Springer Verlag Deutschland (2002)}
   \small{\bibitem{techno1}F.Bitzer/K.M.Brisch: \textit{Digitale Signatur}, Seite 17 bis 24, Springer Verlag Deutschland (1999)}
   \small{\bibitem{techno4}F.Bitzer/K.M.Brisch: \textit{Digitale Signatur}, Seite 46 bis 52, Springer Verlag Deutschland (1999)}

\end{thebibliography}
\printbibliography[keyword={internet},title={Internetseiten?}]
\printbibliography[keyword={buch},title={Literatur}]

\end{document}