Im ersten Schritt wurde die Anbindung der neu zuschaffenden Schnittstelle zur Erfassung der Daten aus dem Signatur-Tablet sowie die Weitergabe und Integration in den bestehenden Datenbestand der Sage Office Line durch FZP geprüft. Im zweiten Schritt wurden verschiedene Anbieter von Signatur-Tablets geprüft. Hier hat sich die Firma Wacom schnell heraus kristallisiert, wegen der Unternehmenserfahrungen seit 1983 \cite{konzept1} sowie der guten Anbindungen des grafischen Tablets mittels des bereit gestellten SDK\footnote{\label{foot:4} Software Development Kit: Sammlung von Werkzeugen und Anwendungen, um eine Software zu erstellen, meist inklusive Dokumentation. \cite{SDK}}. Nach Abschluss der Prüfungen wurde mit der Entwicklung der Schnittstelle begonnen.\\
Die Sage Office Line kann mittels DCM (...) angepasst werden.  

Der erste Schritt ist die Einbindung der benötigten DLL\footnote{\label{foot:5} Dynamic Link Library: Dynamische Programmbibliothek für Microsoft Betriebssysteme. \cite{DLL}} in das Entwicklungsprojekt. Die DLL umfassen unteranderen die benötigten Komponenten zur Kommunikation zwischen der Schnittstelle und der Sage Office Line sowie dem Signatur-Tablet. Des Weiteren werden die DLL des Signatur-Tablets zur Erfassung und Prüfung der digitalen Signatur eingebunden. Die 