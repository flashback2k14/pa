Neben der Erstellung eines Fingerabdrucks mittels Hash-Algorithmus, können der digitalen Signatur noch weitere Daten, wie biometrische Charakteristika des Signaturschlüssel-Inhabers angehängt werden. Die Biometrie setzt sich aus den Worten "bios" (Leben) und "métron" (Maßstab) zusammen und "'beschäftigt sich mit Messungen an Lebewesen und den dazu erforderlichen Mess- und Auswertungsverfahren"' \cite{bioMet1}. Ziel ist die Identifizierung sowie die Authentifizierung von Personen anhand deren einzigartiger und konstanter biometrischer Merkmale. Unter biometrischen Merkmalen versteht man unteranderem, dass Papillarmuster der Finger, die Handgeometrie und die Handlinien, die Vermessung des Gesichtes, die Iris- und Retinastruktur, die DNA, der Stimmabgleich, das Bewegungsmuster und die Unterschrift. Bei einer elektronischen Unterschrift mit biometrischen Merkmalen benötigt man hauptsächlich die Schreibgeschwindigkeit, den ausgeübten Druck, den Bereich und die Dauer der Unterschrift. Das Schriftbild dient ebenfalls zur Identifizierung und Authentifizierung einer Person. Andererseits ist eine 100 prozentige Genauigkeit bei biometrischen Merkmalen nicht gewährleistet. Zu Fehlern in den Mess- und Auswertungsverfahren kann es durch altersbedingte Veränderungen, Körperveränderungen, wie z. B. Verletzung oder Verlust von Gliedmaßen, der Veränderung der Iris kommen. Eine weitere Fehlerquelle entsteht durch Täuschung oder Manipulation von Prüfungsgeräten. \cite{bioMet2}\cite{bioMet3}