Neben der Erstellung eines Fingerabdrucks mittels Hash-Algorithmus, können der digitalen Signatur noch weitere Daten, wie biometrische Charakteristika des Signaturschlüssel-Inhabers angehängt werden. Die Biometrie setzt sich aus den Worten "bios" (Leben) und "métron" (Maßstab) zusammen und beschäftigt sich mit Messungen an Lebewesen und den dazu erforderlichen Mess- und Auswertungsverfahren. \cite{bioMet1} Ziel ist die Identifizierung sowie die Authentifizierung von Menschen anhand deren einzigartiger sowie konstanter biometrischer Merkmale. Unter biometrischen Merkamelen versteht man unteranderem, dass Papillarmuster der Finger, die Handgeometrie und die Handlinien, die Vermessung des Gesichtes, die Iris- und Retinastruktur, die DNA, der Stimmabgleich, das Bewegungsmuster sowie die Unterschrift. Bei einer digitalen Signatur mit biometrischen Merkmalen benötigt man hauptsächlich die Schreibgeschwindigkeit, den ausgeübten Druck, den Bereich und die Dauer der Unterschrift sowie das Schriftbild zur Identifizierung und Authentifizierung einer Person. Jedoch ist eine 100 prozentige Genauigkeit bei biometrischen Merkmalen nicht gewährleistet. Zu Fehlern in den Mess- und Auswertungsverfahren kann es durch altersbedingte Veränderungen, Körperveränderungen, wie z. B. Verletzung oder Verlust von Gliedmaßen, der Veränderung der Iris sowie durch Täuschung oder Manipulation von Prüfungsgeräte kommen. \cite{bioMet2}\cite{bioMet3}