Laut \S 2 Nr. 9 SigG kann ein Nutzer bzw. Antragsteller auf die Ausfertigung einer digitalen Signatur nur eine natürliche Person sein. Aus diesem Grund kann eine juristische Person, wie eine Gesellschaft mit beschränkter Haftung oder eine Aktiengesellschaft, welche durch mehrere natürliche Personen, wie dem Aufsichtsrat, Vorstand oder mehreren Geschäftsführeren, vertreten werden kann, keine eindeutige Identifikation und Authentikation ermöglichen und somit nicht als Nutzer von digitalen Signaturen direkt auftreten. Jedoch besteht die Möglichkeite, dass eine natürliche Person die Nutzung seines Signaturschlüssels für weitere natürliche Personen autorisieren kann ?und somit zum Beispiel mehrere Geschäftsführer einer GmbH mit einem Signatürschlüssel auftreten können?. Desweiteren ist es dem Signaturschlüssel-Inhabers möglich zusätzliche Informationen gemäß \S 5 Abs. 2 SigG in Form eines Attribut-Zertifikates zu hinterlegen. In einem Attibut-Zertifikat können Vertretungsvollmachten für dritte Personen, berufsrechtliche Zulassung und sonstige Angaben zur Person eingetragen werden. Außerdem besteht eine Verpflichtung gegenüber dem Nutzer von digitalen Signaturen, den privaten Schlüssel, meistens auf Chipkarten vor Veränderungen geschützt, sorgfältig aufzubewahren um den Gebrauch und die damit einhergehende Identitätsannahme des Signaturschlüssel-Inhabers von unbefugten Personen zu verhindern. \cite{standdeswissens3}\cite{standdeswissens4}