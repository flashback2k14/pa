Laut \S 2 Nr. 9 SigG kann ein Nutzer bzw. Antragsteller auf die Ausfertigung einer digitalen Signatur nur eine natürliche Person sein. Eine juristische Person in Form einer Gesellschaft mit beschränkter Haftung (GmbH) oder einer Aktiengesellschaft (AG) darf dadruch nicht als Nutzer von digitalen Signaturen direkt auftreten. Der Grund hierfür ist, dass eine GmbH oder eine AG durch mehrere natürliche Personen, wie dem Aufsichtsrat, Vorstand oder mehreren Geschäftsführern, vertreten werden kann. Eine eindeutige Identifikation und Authentikation des Nutzers der digitalen Signatur ist infolgedessen nicht mehr möglich. Andererseits besteht die Möglichkeit, dass eine natürliche Person die Nutzung seines Signaturschlüssels für weitere natürliche Personen autorisiert und entsprechend des Signaturgesetzes beispielsweise mehrere Geschäftsführer einer GmbH mit einem Signaturschlüssel auftreten können. Des Weiteren ist es dem Signaturschlüssel-Inhaber möglich zusätzliche Informationen gemäß \S 5 Abs. 2 SigG in Form eines Attribut-Zertifikates zu hinterlegen. In einem Attribut-Zertifikat können Vertretungsvollmachten für dritte Personen, berufsrechtliche Zulassung und sonstige Angaben zur Person eingetragen werden. Außerdem besteht eine Verpflichtung gegenüber dem Nutzer von digitalen Signaturen, den privaten Schlüssel sorgfältig aufzubewahren. Die Aufbewahrung ist wichtig um den Gebrauch und die damit einhergehende Identitätsannahme des Signaturschlüssel-Inhabers von unbefugten Personen zu verhindern. Der private Schlüssel ist meistens auf einer Chipkarte hinterlegt und vor Veränderungen geschützt. \cite{standdeswissens3}\cite{standdeswissens4}