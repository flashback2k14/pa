\subsubsection*{Einleitung}
Das mittelständige Unternehmen ist in den Bereichen Maschinenbau, Reparaturen und Instandsetzungen sowie Verkauf und Vermietung tätig. Die Produkte des Verkaufs und der Vermietung sind hochwertige Elektromaschinen sowie Werkzeuge und Verbrauchsmaterialien. Desweiteren werden Instandhaltungs- und Reparaturauftäge für Maschinen sowie Spezialanfertigung im Bereich Maschinenbau angeboten. Die Kundschaft besteht überwiegend aus selbstständigen Handwerkern, kleineren Betrieben und Angehörigen aus Land- und Forstwirtschaft.
\subsubsection*{Ist-Zustand}
Die Warenabgabe erfolgt mittels handgeschriebenen Rechnungen und Aufträgen. Im nächsten Schritt werden die Aufträge und Rechnungen zur weiteren Verwendung in das bestehende MSDOS-System händisch übertragen. Im darauf folgenden Schritt werden die handgeschriebenen Belege an die Buchhaltungsabteilung sowie an die Fertigungsabteilung weiter gereicht zur Auftragsbearbeitung. Aufgrund der zunehmenden Aufträge und Komplexität ist das eingesetzte System nicht mehr effektiv und fehleranfällig. Aus diesem Grund ist eine Softwareumstellung auf ein vollwertiges ERP System geplant. Im Zuge der Umstellung wurde von dem mittelständigen Unternehmen die Schaffung der personalisierten Warenabgabe mittels digitaler Signatur gewünscht.
\subsubsection*{Ziel}
Das Hauptziel der Umstellung ist die Vereinfachung und Beschleunigung der Geschäftsprozesse sowie der Kostenreduzierung und Effizienzsteigerung. Das Ziel der personalisierten Warenabgabe ist der eindeutige Nachweis über die erbrachten Lieferungen und Leistungen sowie der rechtlichen Absicherung. \cite{einleitung1}