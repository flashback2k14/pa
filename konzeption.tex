Motivation:\\
--Ausgangssituation (Projektumfeld, Probleme) --> Bereits in Einleitung!?\\
--Warum das Thema? Warum dieses Projekt?\\\\
Ziele:\\
--Was erreichen und verändern? (gedanken, keine Schritte)\\\\
Zielgruppe:\\
--Beschreibung\\
--Hauptzielgruppe?\\
--Wer profitiert davon?\\\\
Ort:\\
--Ort, Region,Besonderheiten\\\\
Umsetzung:\\
--konkrete Schritte Erfüllung Ziele\\
--nicht jedes kleine Detail notwendig\\
--Was wird unternommen um Ziele zu erreichen?\\\\
Zeitplan:\\
--Beginn / Ende, Meilensteine\\
--Orientierung an Umsetzungsschritte\\\\
Wirkung: --> evtl. nicht mit berücksichtigen\\
--visionäre Gedanken\\
--Welche Folgen hat Projekt?\\\\
\url{http://www.phase0.org/wp-content/uploads/Musterkonzept.pdf}\\\\
%xxxxxxxxxxxxxxxxxxxxxxxxxxxxxxx
das mittelständige unternehmen ist in den bereichen maschinenbau, reparaturen, instandsetzungen sowie verkauf und vermietung tätig. wie eingangs erwähnt ist das eingesetzte MSDOS system den anforderungen und kapazitäten nicht mehr gewachsen. aus diesem grund ist eine softwareumstellung auf ein vollwertiges ERP system geplant sowie die schaffung der personalisierten warenabgabe mittels digitaler signatur.\\\\
zur zeit läuft die warenabgabe wie folgt, \textbf{sollkonzept}... Die hauptprodukte des verkaufs bzw. vermietung sind hochwertige maschinen, \textbf{SK}... Die Hauptabnehmer sind \textbf{SK}, landwirte... \cite{einleitung1}

