\subsubsection*{Einleitung}
Das mittelständige Unternehmen ist in den Bereichen Maschinenbau, Reparaturen und Instandsetzungen sowie Verkauf und Vermietung tätig. Die Produkte des Verkaufs und der Vermietung sind hochwertige Elektromaschinen sowie Werkzeuge und Verbrauchsmaterialien. Desweiteren werden Instandhaltungs- und Reparaturaufträge für Maschinen sowie Spezialanfertigung im Bereich Maschinenbau angeboten. Die Kundschaft besteht überwiegend aus selbstständigen Handwerkern, kleineren Betrieben und Angehörigen aus Land- und Forstwirtschaft.
\subsubsection*{Ist-Zustand}
Die Warenabgabe erfolgt mittels handgeschriebener Rechnungen und Aufträge. Im nächsten Schritt werden die Aufträge und Rechnungen zur weiteren Verwendung in das bestehende MSDOS-System wiederholt händisch übertragen. Im darauffolgenden Schritt werden die handgeschriebenen Belege an die Buchhaltungsabteilung zur Prüfung des Rechnungseingangs und Rechnungsausgangs sowie an die Fertigungsabteilung zur Auftragsbearbeitung weitergereicht.
\subsubsection*{Ziel}
Das Hauptziel der Umstellung ist die Vereinfachung und Beschleunigung der Geschäftsprozesse sowie der Kostenreduzierung und Effizienzsteigerung. Das Ziel der personalisierten Warenabgabe ist der eindeutige Nachweis über die erbrachten Lieferungen und Leistungen sowie der rechtlichen Absicherung gegenüber Dritten. \cite{einleitung1}
\subsubsection*{Umsetzung}
Im weiteren Verlauf des Konzeptes werde ich nur auf die Umsetzung der personalisierten Warenabgabe mittels digitaler Signatur eingehen. Die Erfassung der digitalen Signatur wird mittels grafischen Tablets der Firma Wacom realisiert. Aufgrund der langjährigen Unternehmenserfahrungen sowie der guten Anbindungen des grafischen Tablets mittels des bereit gestellten SDK\footnote{\label{foot:4} Software Development Kit: Sammlung von Werkzeugen und Anwendungen, um eine Software zu erstellen, meist inklusive Dokumentation. \cite{SDK}} fiel die Entscheidung zur Nutzung der Produkte der Firma Wacom. Desweiteren zeichnete sich die Firma durch einen guten Support aus.  