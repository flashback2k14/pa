Der Ausgangspunkt für die Auseinandersetzung mit dem Thema "'Biometrische Unterschrift"' war ein Kundenauftrag. Das Ziel des Kundenauftrags ist die Einführung einer personalisierte Warenabgabe im Bereich Verkauf und Vermietung mittels der Schaffung eines Zusatzmoduls zwischen der Sage Office Line und einem Signatur-Tablet.
\newline
Die Anbindung einer Erweiterung ist hinsichtlich des modularen Aufbaus der Sage Office Line möglich. Die verwendete Technologie in der Softwareentwicklung basiert auf dem Microsoft .NET Framework in Verbindung mit Microsoft Access und dem Microsoft COM Framework. Die Verwaltung der Daten wurde mit einem Microsoft SQL Server realisiert. Die verwendete Technik im Hardwarebereich ist ein Signatur-Tablet der Firma Wacom. Der Einsatz der biometrischen Unterschrift ist angesichts des sofort verfügbaren Nachweises über erbrachte Lieferungen gewählt wurden. Mit der biometrischen Unterschrift werden Daten, wie die Schreibgeschwindigkeit, der ausgeübten Druck, der Bereich und die Dauer der Unterschrift sowie das Schriftbild zur Identifizierung vom Unterzeichner gesammelt und gespeichert. Eine biometrische Unterschrift ist rechtlich bindend und in gerichtlichen Verfahren als Beweismittel anerkannt. 
\newline
Das Ergebnis ist eine voll integrierte Erweiterung in die Sage Office Line. Ein weiteres Ergebnis der Entwicklung ist das wiederverwendbare Steuerelement zur Erfassung der biometrischen Unterschrift. Das Steuerelement kann dadurch in weiteren Zusatzmodulen und Erweiterungen verwendet werden. Ein weiterer Einsatzort wäre in der Lagerverwaltung möglich. Hier wäre die Schaffung eines Zusatzmoduls im Bereich Lagerhaltung und Kommissionierung zur Erfassung der personalisierten Ein- und Auslagerung von Waren und Vermietungsgegenständen denkbar.