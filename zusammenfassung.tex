Der Ausgangspunkt für die Auseinandersetzung mit dem Thema "'Digitale Signatur"' war ein Kundenauftrag. Das Ziel des Kundenauftrags ist die Einführung einer personalisierte Warenabgabe im Bereich Verkauf und Vermietung mittels der Schaffung einer Schnittstelle zwischen der Sage Office Line und einem Signatur-Tablet.
\newline
Die Anbindung einer Schnittstelle ist hinsichtlich des modularen Aufbaus der Sage Office Line möglich. Die verwendete Technologie in der Softwareentwicklung basiert auf dem Microsoft Visual Basic .NET-Framework in Verbindung mit dem Microsoft COM-Framework. Die Verwaltung der Daten wurde mit einem Microsoft SQL-Server realisiert. Die verwendete Technik im Hardwarebereich ist ein Signatur-Tablet der Firma Wacom. Der Einsatz der digitalen Signatur ist angesichts des sofort verfügbaren Nachweises über erbrachte Lieferungen und Leistungen gewählt wurden. In der digitalen Signatur werden Daten, wie die Schreibgeschwindigkeit, der ausgeübten Druck, der Bereich und die Dauer der Unterschrift sowie das Schriftbild zur Identifizierung vom Unterzeichner gesammelt und gespeichert. Eine digitale Signatur ist rechtlich bindend und in gerichtlichen Verfahren als Beweismittel anerkannt. 
\newline
Das Ergebnis ist eine voll integrierte Schnittstelle in die Sage Office Line sowie ein zufriedener Kunde. Ein weiteres Ergebnis der Entwicklung der Schnittstelle ist das geschaffene, wiederverwendbare Steuerelement zur Erfassung der elektronischen Unterschrift. Das Steuerelement kann dadurch in weiteren Schnittstellen verwendet werden. Ein weiterer Einsatzort wäre in der Lagerverwaltung möglich. Hier wäre die Schaffung einer Schnittstelle im Bereich Lagerhaltung \& Kommissionierung zur Erfassung der personalisierten Ein- und Ausgabe von Lagergegenständen denkbar.